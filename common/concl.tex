%% Согласно ГОСТ Р 7.0.11-2011:
%% 5.3.3 В заключении диссертации излагают итоги выполненного исследования, рекомендации, перспективы дальнейшей разработки темы.
%% 9.2.3 В заключении автореферата диссертации излагают итоги данного исследования, рекомендации и перспективы дальнейшей разработки темы.
\begin{enumerate}
  \item Проведен анализ существующих средств концептуального моделирования гипотез в процессе проведения виртуальных экспериментов. На основе анализа \ldots
  \item Осуществлена разработка методов автоматизированного построения решеток гипотез в виртуальных экспериментах. \ldots
  \item Проведена реализация метода построения решеток гипотез в виде отдельного компонента системы. \ldots
  \item Проведен анализ методов формального манипулирования виртуальными экспериментами и гипотезами при сохранении непротиворечивости экспериментов для систем с явным представлением и использованием гипотез.
  \item Проведено исследование и разработка методов автоматического порождения гипотез из данных.
  \item Разработаны метод поиска оптимальных параметров гипотез и метод корректной оценки соответствия гипотезы исследуемому явлению.
  \item Реализован метод повышения эффективности и скорости проведения виртуальных экспериментов.
  \item Применимость разработанных методов и подходов продемонстрирована на задаче из предметной области нейрофизиологии, связанной с поиском функциональной связности областей человеческого мозга над данными проекта HCP.
  \item Разработаны методы сравнения построенных из данных гипотез с теоретическими гипотезами из теоретического моделирования или литературы.
  \item Разработан программный прототип по обеспечению полного цикла работы с виртуальным экспериментом.
  
  \item Разработаны программы, реализующие конкретные виртуальные эксперименты в ОИИД. Для выполнения поставленных задач был создан \ldots
\end{enumerate}

Поставленные задачи диссертационного исследования выполнены полностью.

Результаты диссертации были применены в ходе выполнения научно-исследовательских работ:
\begin{enumerate}
    \item Методы и средства организации экспериментов в движимых гипотезами исследованиях в областях с интенсивным использованием данных Грант РФФИ 18-07-01434 (A).
\end{enumerate}