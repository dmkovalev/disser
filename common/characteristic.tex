
{\actuality} В настоящее время в большинстве научных областей произошел значительный скачок в количестве производимых 
и накопленных данных, получаемых в результате наблюдений, экспериментов или компьютерного моделирования. Во многих 
научных областях работа с данными стала занимать значительную часть рабочего времени исследователей. Огромные объемы 
данных, получаемых от научных приборов, датчиков, моделей, а также в результате накопления данных в Интернете и 
социальных сетях изменили сам формат и подход к научной деятельности, во главу угла была поставлена работа с данными. 
Это привело к возникновению новой парадигмы в науке, называемой Четвертой парадигмой, в рамках которой проводятся 
исследования с интенсивным использованием данных (ИИИД). Примерами научных направлений с ИИИД являются астрономия, 
нейрофизиология, науки о Земле, биология. 

Фундаментальная цель ИИИД "--- получение (вывод) знаний на основе совокупных данных, организованных в сетевые 
инфраструктуры (как, например, хранилища данных, вычислительные грид-системы, облачные базы данных). В рамках 
ИИИД изучаются конкретные концептуально моделирующие средства, дающие возможность ученым представлять научные 
гипотезы, модели и связанные с ними вычислительные интерпретации, которые можно сравнить с наблюдениями явлений. 
Модель позволяет ученым фиксировать существующее знание относительно наблюдаемого и исследуемого явления, включая 
его формальную математическую интерпретацию, если она существует. Декларативное представление научной модели дает 
возможность ученым сосредоточиться на исследовании научных вопросов. Для преодоления разрыва между онтологическим 
описанием исследуемого явления и его симуляциями могут быть также использованы гипотезы. Концептуальные воззрения 
относительно сущностей области науки позволяют осуществлять поиск определений, поддерживающих распространение 
научных моделей между разными научными группами. 

Количество работ, рассматривающих гипотезы, эксперименты и модели, увеличивается. В этих работах 
рассматривается определение гипотез в явном виде, концептуализируются элементы, связанные с исследованиями, 
движимыми гипотезами: наблюдаемое явление, интерпретирующее это явление модель, а также метаданные модели. 
Разрабатываются различные средства и инструменты для работы с гипотезами: их генерации, оценке и проверке, 
сравнении нескольких гипотез между собой. Вопросами развития движимых гипотезами исследований в различных 
областях с интенсивным использованием данных занимались многие отечественные и зарубежные ученые, такие 
как Л.~A. Калиниченко, R.~D. King , M. Liakata, C. Lu, S.G. Oliver, L.N. Soldatova., J. Duggan, M. Brodie, 
F. Porto, B. Gonçalves и другие. 

Однако, существующие системы не охватывают несколько важных аспектов, в том числе их формальной спецификации, 
использования гипотез, добавления новых гипотез в эксперимент, взаимодействия между несколькими гипотезами в 
одном эксперименте, отслеживание эволюции эксперимента, эффективное планирование эксперимента. 

\textit{Таким образом, существует необходимость разработки новых методов управления виртуальным экспериментом 
 в гипотезо-ориентированных ИИИД, а также систем для проведения таких виртуальных экспериментов.}
 
Основными задачами является разработка представления виртуальным экспериментом, в т.ч. описания его основных 
элементов, жизненного цикла, разработки методов построения зависимостей между несколькими гипотезами в одном 
эксперименте и добавления новых гипотез в такие структуры. Далее требуется разработать платформу 
проведения виртуальных экспериментов, а также ускорить его проведение за счет вычислительных оптимизаций, 
используемых при расчетах. Важной задачей является оценка вычислительной эффективности построения структур гипотез. 
При повторном расчете виртуальных экспериментов важной задачей является разработка 
метода повторного использования части уже вычисленных фрагментов гипотез. 
Разработанные методы и подходы позволят 
усовершенствовать их применение в гипотезо-ориентированных системах для решения научных задач 
в конкретных областях с ИИИД.

Основные результаты диссертации были получены в процессе выполнения работ по научному гранту 
РФФИ 18-07-01434 А "--- <<Методы и средства организации экспериментов в движимых гипотезами исследованиях 
в областях с интенсивным использованием данных>>.


% {\progress}
% Этот раздел должен быть отдельным структурным элементом по
% ГОСТ, но он, как правило, включается в описание актуальности
% темы. Нужен он отдельным структурынм элемементом или нет ---
% смотрите другие диссертации вашего совета, скорее всего не нужен.

{\aim} данной работы является исследование и разработка методов и средств поддержки научных 
исследований для обеспечения повторного использования научных методов и воспроизводимости результатов 
их работы в движимых гипотезами исследованиях с интенсивным использованием данных.


Для~достижения поставленной цели необходимо было решить следующие {\tasks}:
\begin{enumerate}[beginpenalty=10000] % https://tex.stackexchange.com/a/476052/104425
    \item Разработать метод управления виртуальным экспериментом в ИИИД, в т.ч. метод построение решеток 
            гипотез в виртуальных экспериментах, метод повышения эффективности проведения виртуальных 
            экспериментов, методы сравнения построенных из данных гипотез, исследовать их эффективность.
    \item Разработать комплекс программ, позволяющий обеспеченить полный цикл работ с виртуальным экспериментом, 
            реализующей разработанные методы и архитектуру.
    %\item Разработать методы расширения виртуального эксперимента новой гипотезой при сохранении непротиворечивости оригинального эксперимента, а также ее интеграции с ранее полученными данными.
    \item Разработать методику работы с гипотезо-ориентированным виртуальным экспериментом с демонстрацией 
            применимости на конкретных задачах в области нейрофизиологии.
    %\item Исследовать, разработать, вычислить и~т.\:д. и~т.\:п.
\end{enumerate}


{\novelty}
\begin{enumerate}[beginpenalty=10000] % https://tex.stackexchange.com/a/476052/104425
  \item Предложен новый метод управления виртуальным эксприментом в ИИИД, позволяющий 
  
        построения решеток гипотез в виртуальных экспериментах по  \ldots
  \item Впервые \ldots
  \item Было выполнено оригинальное исследование \ldots
\end{enumerate}

\textbf{Соответствие диссертации паспорту научной специальности.}
В соответствии с формулой специальности 2.3.5 <<Математическое и программное обеспечение вычислительных систем, 
комплексов и компьютерных сетей (технические науки)>> в работе выполнены исследование, разработка, и реализация 
методов и программного комплекса, обеспивающего полный цикл работ с виртуальным экспериментом, 
реализующей разработанные методы и архитектуру. Работа соответствует следующим пунктам паспорта специальности: 
п. 1 <<Модели, методы и алгоритмы проектирования, анализа, трансформации,
верификации и тестирования программ и программных систем>>, п. 3 <<Модели, методы, архитектуры, алгоритмы, 
языки и программные инструменты организации взаимодействия программ и программных систем>>, п.10 <<Оценка качества, 
стандартизация и сопровождение программных систем>>.

Реализация эффективных численных методов и алгоритмов в виде комплексов
проблемно-ориентированных программ для проведения вычислительного
эксперимента», п. 7 «Качественные или аналитические методы исследования математических моделей (технические науки)» и п. 9 «Постановка и
проведение численных экспериментов, статистический анализ их результатов, в том числе с применением современных компьютерных технологий
(технические науки)»

{\influence} \ldots

{\methods} \ldots

{\defpositions}
\begin{enumerate}[beginpenalty=10000] % https://tex.stackexchange.com/a/476052/104425
  \item Первое положение
  \item Второе положение
  \item Третье положение
  \item Четвертое положение
\end{enumerate}
В папке Documents можно ознакомиться с решением совета из Томского~ГУ
(в~файле \verb+Def_positions.pdf+), где обоснованно даются рекомендации
по~формулировкам защищаемых положений.

{\reliability} полученных результатов обеспечивается \ldots \ Результаты находятся в соответствии с результатами, полученными другими авторами.


{\probation}
Основные результаты работы докладывались~на:
перечисление основных конференций, симпозиумов и~т.\:п.

{\contribution} Автор принимал активное участие \ldots

\ifnumequal{\value{bibliosel}}{0}
{%%% Встроенная реализация с загрузкой файла через движок bibtex8. (При желании, внутри можно использовать обычные ссылки, наподобие `\cite{vakbib1,vakbib2}`).
    {\publications} Основные результаты по теме диссертации изложены
    в~XX~печатных изданиях,
    X из которых изданы в журналах, рекомендованных ВАК,
    X "--- в тезисах докладов.
}%
{%%% Реализация пакетом biblatex через движок biber
    \begin{refsection}[bl-author, bl-registered]
        % Это refsection=1.
        % Процитированные здесь работы:
        %  * подсчитываются, для автоматического составления фразы "Основные результаты ..."
        %  * попадают в авторскую библиографию, при usefootcite==0 и стиле `\insertbiblioauthor` или `\insertbiblioauthorgrouped`
        %  * нумеруются там в зависимости от порядка команд `\printbibliography` в этом разделе.
        %  * при использовании `\insertbiblioauthorgrouped`, порядок команд `\printbibliography` в нём должен быть тем же (см. biblio/biblatex.tex)
        %
        % Невидимый библиографический список для подсчёта количества публикаций:
        \printbibliography[heading=nobibheading, section=1, env=countauthorvak,          keyword=biblioauthorvak]%
        \printbibliography[heading=nobibheading, section=1, env=countauthorwos,          keyword=biblioauthorwos]%
        \printbibliography[heading=nobibheading, section=1, env=countauthorscopus,       keyword=biblioauthorscopus]%
        \printbibliography[heading=nobibheading, section=1, env=countauthorconf,         keyword=biblioauthorconf]%
        \printbibliography[heading=nobibheading, section=1, env=countauthorother,        keyword=biblioauthorother]%
        \printbibliography[heading=nobibheading, section=1, env=countregistered,         keyword=biblioregistered]%
        \printbibliography[heading=nobibheading, section=1, env=countauthorpatent,       keyword=biblioauthorpatent]%
        \printbibliography[heading=nobibheading, section=1, env=countauthorprogram,      keyword=biblioauthorprogram]%
        \printbibliography[heading=nobibheading, section=1, env=countauthor,             keyword=biblioauthor]%
        \printbibliography[heading=nobibheading, section=1, env=countauthorvakscopuswos, filter=vakscopuswos]%
        \printbibliography[heading=nobibheading, section=1, env=countauthorscopuswos,    filter=scopuswos]%
        %
        \nocite{*}%
        %
        {\publications} Основные результаты по теме диссертации изложены в~\arabic{citeauthor}~печатных изданиях,
        \arabic{citeauthorvak} из которых изданы в журналах, рекомендованных ВАК\sloppy%
        \ifnum \value{citeauthorscopuswos}>0%
            , \arabic{citeauthorscopuswos} "--- в~периодических научных журналах, индексируемых Web of~Science и Scopus\sloppy%
        \fi%
        \ifnum \value{citeauthorconf}>0%
            , \arabic{citeauthorconf} "--- в~тезисах докладов.
        \else%
            .
        \fi%
        \ifnum \value{citeregistered}=1%
            \ifnum \value{citeauthorpatent}=1%
                Зарегистрирован \arabic{citeauthorpatent} патент.
            \fi%
            \ifnum \value{citeauthorprogram}=1%
                Зарегистрирована \arabic{citeauthorprogram} программа для ЭВМ.
            \fi%
        \fi%
        \ifnum \value{citeregistered}>1%
            Зарегистрированы\ %
            \ifnum \value{citeauthorpatent}>0%
            \formbytotal{citeauthorpatent}{патент}{}{а}{}\sloppy%
            \ifnum \value{citeauthorprogram}=0 . \else \ и~\fi%
            \fi%
            \ifnum \value{citeauthorprogram}>0%
            \formbytotal{citeauthorprogram}{программ}{а}{ы}{} для ЭВМ.
            \fi%
        \fi%
        % К публикациям, в которых излагаются основные научные результаты диссертации на соискание учёной
        % степени, в рецензируемых изданиях приравниваются патенты на изобретения, патенты (свидетельства) на
        % полезную модель, патенты на промышленный образец, патенты на селекционные достижения, свидетельства
        % на программу для электронных вычислительных машин, базу данных, топологию интегральных микросхем,
        % зарегистрированные в установленном порядке.(в ред. Постановления Правительства РФ от 21.04.2016 N 335)
    \end{refsection}%
    \begin{refsection}[bl-author, bl-registered]
        % Это refsection=2.
        % Процитированные здесь работы:
        %  * попадают в авторскую библиографию, при usefootcite==0 и стиле `\insertbiblioauthorimportant`.
        %  * ни на что не влияют в противном случае
        \nocite{vakbib2}%vak
        \nocite{patbib1}%patent
        \nocite{progbib1}%program
        \nocite{bib1}%other
        \nocite{confbib1}%conf
    \end{refsection}%
        %
        % Всё, что вне этих двух refsection, это refsection=0,
        %  * для диссертации - это нормальные ссылки, попадающие в обычную библиографию
        %  * для автореферата:
        %     * при usefootcite==0, ссылка корректно сработает только для источника из `external.bib`. Для своих работ --- напечатает "[0]" (и даже Warning не вылезет).
        %     * при usefootcite==1, ссылка сработает нормально. В авторской библиографии будут только процитированные в refsection=0 работы.
}

При использовании пакета \verb!biblatex! будут подсчитаны все работы, добавленные
в файл \verb!biblio/author.bib!. Для правильного подсчёта работ в~различных
системах цитирования требуется использовать поля:
\begin{itemize}
        \item \texttt{authorvak} если публикация индексирована ВАК,
        \item \texttt{authorscopus} если публикация индексирована Scopus,
        \item \texttt{authorwos} если публикация индексирована Web of Science,
        \item \texttt{authorconf} для докладов конференций,
        \item \texttt{authorpatent} для патентов,
        \item \texttt{authorprogram} для зарегистрированных программ для ЭВМ,
        \item \texttt{authorother} для других публикаций.
\end{itemize}
Для подсчёта используются счётчики:
\begin{itemize}
        \item \texttt{citeauthorvak} для работ, индексируемых ВАК,
        \item \texttt{citeauthorscopus} для работ, индексируемых Scopus,
        \item \texttt{citeauthorwos} для работ, индексируемых Web of Science,
        \item \texttt{citeauthorvakscopuswos} для работ, индексируемых одной из трёх баз,
        \item \texttt{citeauthorscopuswos} для работ, индексируемых Scopus или Web of~Science,
        \item \texttt{citeauthorconf} для докладов на конференциях,
        \item \texttt{citeauthorother} для остальных работ,
        \item \texttt{citeauthorpatent} для патентов,
        \item \texttt{citeauthorprogram} для зарегистрированных программ для ЭВМ,
        \item \texttt{citeauthor} для суммарного количества работ.
\end{itemize}
% Счётчик \texttt{citeexternal} используется для подсчёта процитированных публикаций;
% \texttt{citeregistered} "--- для подсчёта суммарного количества патентов и программ для ЭВМ.

Для добавления в список публикаций автора работ, которые не были процитированы в
автореферате, требуется их~перечислить с использованием команды \verb!\nocite! в
\verb!Synopsis/content.tex!.
