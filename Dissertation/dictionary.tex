\chapter*{Словарь терминов}             % Заголовок
\addcontentsline{toc}{chapter}{Словарь терминов}  % Добавляем его в оглавление

\textbf{Научная гипотеза} : есть предлагаемое объяснение явления, которое еще должно быть подвергнуто строгой проверке 

\textbf{Научная теория} : уже прошла всесторонние испытания и широко принята в качестве точного объяснения, стоящего за наблюдением

\textbf{Научный закон} : это утверждение, которое показывает некоторую упорядоченность или регулярность в природе, существование неизменной связи между определенным комплексом условий и определенными явлениями

\textbf{Индукция} : сбор и интерпретация эмпирических доказательств. Это – методика, в которой собираются и рассматриваются разрозненные элементы доказательства до того момента, когда открывается закон или изобретается теория

\textbf{Научная теория} : обобщает гипотезу или группу гипотез, поддержанных неоднократными проверками. Теория справедлива, если нет ни одного противоречащего ей факта

\textbf{Научная парадигма} : объясняет рабочее множество теорий, на которых основана наука

\textbf{Решетка гипотез} : образуется при рассмотрении множества гипотез, расположенных в строгом порядке 

\textit{wasDerivedFrom} : была выведена из (снизу вверх). Гипотезы, непосредственно выведенные из одной-единственной гипотезы, называются атомарными, в то время как те, которые выведены, по крайней мере, из двух гипотез, называются комплексными

\textbf{Формульное моделирование} : это процесс оценки соотношений между переменными